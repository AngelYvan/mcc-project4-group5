%package list
\documentclass{article}
\usepackage[top=3cm, bottom=3cm, outer=3cm, inner=3cm]{geometry}
\usepackage{graphicx}
\usepackage{url}


%% \usepackage{cite}
\usepackage{hyperref}
\usepackage{array}
\usepackage{multicol}
\newcolumntype{x}[1]{>{\centering\arraybackslash\hspace{0pt}}p{#1}}
\usepackage{natbib}
\usepackage{pdfpages}
\usepackage{multirow}
\usepackage{float}
\usepackage[normalem]{ulem}
\useunder{\uline}{\ul}{}
\usepackage{listings}


\usepackage[utf8]{inputenc}
\usepackage[T1]{fontenc}
\usepackage[spanish]{babel}
\usepackage{times}

%%%%%%%%%%%%%%%%%%%%%%%%%%%%%%%%%%%%%%%%%%%%%%%%%%%%%%%%%%%%%%%%%%%%%%%%%%%%
%%%%%%%%%%%%%%%%%%%%%%%%%%%%%%%%%%%%%%%%%%%%%%%%%%%%%%%%%%%%%%%%%%%%%%%%%%%%
\newcommand{\csemail}{vmachacaa@unsa.edu.pe}
\newcommand{\csdocente}{Vicente Machaca Arceda}
\newcommand{\cscurso}{Algoritmos y Estructura de Datos}
\newcommand{\csuniversidad}{Universidad Nacional de San Agustín}
\newcommand{\csescuela}{Maestría en Ciencia de la Computación}
\newcommand{\cspracnr}{04}
\newcommand{\cstema}{--}
%%%%%%%%%%%%%%%%%%%%%%%%%%%%%%%%%%%%%%%%%%%%%%%%%%%%%%%%%%%%%%%%%%%%%%%%%%%%
%%%%%%%%%%%%%%%%%%%%%%%%%%%%%%%%%%%%%%%%%%%%%%%%%%%%%%%%%%%%%%%%%%%%%%%%%%%%


\usepackage[english,spanish]{babel}
\usepackage[utf8]{inputenc}
\AtBeginDocument{\selectlanguage{spanish}}
\renewcommand{\figurename}{Figura}
\renewcommand{\refname}{Referencias}
\renewcommand{\tablename}{Tabla} %esto no funciona cuando se usa babel
\AtBeginDocument{%
	\renewcommand\tablename{Tabla}
}

\usepackage{fancyhdr}
\pagestyle{fancy}
\fancyhf{}
\setlength{\headheight}{30pt}
\renewcommand{\headrulewidth}{1pt}
\renewcommand{\footrulewidth}{1pt}
\fancyhead[L]{\raisebox{-0.2\height}{\includegraphics[width=3cm]{img/logo_unsa.jpg}}}
\fancyhead[C]{}
\fancyhead[R]{\fontsize{7}{7}\selectfont	\csuniversidad \\ \csescuela \\ \textbf{\cscurso} }
\fancyfoot[L]{MSc. Vicente Machaca}
\fancyfoot[C]{\cscurso}
\fancyfoot[R]{Página \thepage}
%%%%%%%%%%%%%%%%%%%%%%%%%%%%%%%%%%%%%%%%%%%%%%%%%%%%%%%%%%%%%%%%%%%%%%%%%%%%%%%%%%%%%%%%%%%%%%%%%%%%%%%%%%%%%%%%%%%%%%%%%%%%%%%%%%%%%%%%%%%%%%%%%%%%%%%%%%%%%%%%%%
% inicio  de codigo 
\usepackage{color}
\definecolor{gray97}{gray}{.97}
\definecolor{gray75}{gray}{.75}
\definecolor{gray45}{gray}{.45}

\usepackage{listings}
\lstset{ frame=Ltb,
framerule=0pt,
aboveskip=0.5cm,
framextopmargin=3pt,
framexbottommargin=3pt,
framexleftmargin=0.4cm,
framesep=0pt,
rulesep=.4pt,
backgroundcolor=\color{gray97},
rulesepcolor=\color{black},
%
stringstyle=\ttfamily,
showstringspaces = false,
basicstyle=\small\ttfamily,
commentstyle=\color{gray45},
keywordstyle=\bfseries,
%
numbers=left,
numbersep=15pt,
numberstyle=\tiny,
numberfirstline = false,
breaklines=true,
}

% minimizar fragmentado de listados
\lstnewenvironment{listing}[1][]
{\lstset{#1}\pagebreak[0]}{\pagebreak[0]}

\lstdefinestyle{consola}
{basicstyle=\scriptsize\bf\ttfamily,
backgroundcolor=\color{gray75},
}


\begin{document}
	
	\vspace*{10px}
	
	\begin{center}	
		\fontsize{17}{17} \textbf{ Práctica \cspracnr}
	\end{center}
	%\centerline{\textbf{\underline{\Large Título: Informe de revisión del estado del arte}}}
	%\vspace*{0.5cm}
	

	\begin{table}[h]
		\begin{tabular}{|x{4.7cm}|x{4.8cm}|x{4.8cm}|}
			\hline 
			\textbf{DOCENTE} & \textbf{CARRERA}  & \textbf{CURSO}   \\
			\hline 
			\csdocente & \csescuela & \cscurso    \\
			\hline 
		\end{tabular}
	\end{table}	
	
	
	\begin{table}[h]
		\begin{tabular}{|x{4.7cm}|x{4.8cm}|x{4.8cm}|}
			\hline 
			\textbf{PRÁCTICA} & \textbf{TEMA}  & \textbf{DURACIÓN}   \\
			\hline 
			\cspracnr & Kdtree & 3 horas   \\
			\hline 
		\end{tabular}
	\end{table}
	
	
	\section{Datos de los estudiantes}
	\begin{itemize}
		\item Grupo: V
		\item Integrantes: 
		\begin{itemize}
			\item Angel Yvan Choquehuanca Peraltilla
			\item Estefany Pilar Huaman Colque
            \item Eduardo Diaz Huayhuas
            \item Gustavo Raul Manrique Fernandez
		\end{itemize}		
	\end{itemize}
	
	
 
	
	%\clearpage
	%\bibliographystyle{apalike}
	%\bibliographystyle{IEEEtranN}
	%\bibliography{bibliography}
		
  
\section{Marco Teorico}

\subsection{Concepto}
El árbol k -d es un árbol binario en el que cada nodo es un punto k -dimensional. Se puede pensar que cada nodo que no es una hoja genera implícitamente un hiperplano de división que divide el espacio en dos partes, conocidas como medios espacios . Los puntos a la izquierda de este hiperplano están representados por el subárbol izquierdo de ese nodo y los puntos a la derecha del hiperplano están representados por el subárbol derecho. La dirección del hiperplano se elige de la siguiente manera: cada nodo en el árbol está asociado con uno de los kdimensiones, con el hiperplano perpendicular al eje de esa dimensión. Entonces, por ejemplo, si para una división particular se elige el eje "x", todos los puntos en el subárbol con un valor "x" más pequeño que el nodo aparecerán en el subárbol izquierdo y todos los puntos con un valor "x" más grande aparecerán. estar en el subárbol derecho. En tal caso, el hiperplano estaría fijado por el valor x del punto, y su normal sería el eje x unitario.

\subsection{Particionado}
Un árbol kd emplea sólo planos perpendiculares a uno de los ejes del sistema de coordenadas. 
Todos los nodos de un árbol kd, desde el nodo raíz hasta los nodos hoja, almacenan un punto.
Cada plano debe pasar a través de uno de los puntos del árbol kd.

\section{Metodologia y Desarrollo}

Se realizará una app que permita el desarrollo del algoritmo Kdtree suiendo las tareas preestablecidas por la practica 04.

Tarea 1: Cree un archivo main.html.

Tarea 2: Cree un archivo kdtree.js.
\lstdefinestyle{C}
{language=C,
}

\begin{lstlisting}[style=C]
function getHeight(node) {
    if(node == null)   
        return 0;
    else   
        return 1 +(Math.max(getHeight(node.left),getHeight(node.right))); 
}
function generate_dot(node) {
    var cad="";
	if(node==null)
		return "";

	if(node.left!=null)
	{
		cad=cad+'"'+node.point.toString()+"\"";
		cad=cad+" -> "+'"'+node.left.point.toString()+'"'+";"+"\n";
	}
	if(node.right!=null)
	{
		cad=cad+"\""+node.point.toString()+"\"";
		cad=cad+" -> "+'"'+node.right.point.toString()+'"'+";"+"\n";
	}
	return cad+generate_dot(node.left)+generate_dot(node.right);
}
function build_kdtree(points, depth = 0) {
    var n = points.length;
    var axis = depth % k;


    if (n <= 0) {
        return null;
    }
    if (n == 1) {
        return new Node(points[0], axis)
    }

    var median = Math.floor(points.length / 2);

    // sort by the axis
    points.sort(function (a, b) {
        return a[axis] - b[axis];
    });
    //console.log(points);

    var left = points.slice(0, median);
    var right = points.slice(median + 1);

    //console.log(right);

    var node = new Node(points[median].slice(0, k), axis);
    node.left = build_kdtree(left, depth + 1);
    node.right = build_kdtree(right, depth + 1);

    return node;
}


\end{lstlisting}

Tarea 3: Cree un archivo sketch.js y evalue sus resultados.
\begin{lstlisting}[style=C]
function setup() {
    var width = 250;
    var height = 200;
    let kdTreeCanvas = createCanvas(width, height);
    kdTreeCanvas.parent("KdTreeCanvas");
    background(0);
    for (var x = 0; x < width; x += width / 10) {
        for (var y = 0; y < height; y += height / 5) {
            stroke(125, 125, 125);
            strokeWeight(1);
            line(x, 0, x, height);
            line(0, y, width, y);
        }
    }
    var data = [];
    for (let i = 0; i < 12; i++) {
        var x = Math.floor(Math.random() * height);
        var y = Math.floor(Math.random() * height);
        data.push([x, y]);
        fill(255, 255, 255);
        circle(x, height - y, 7); // 200 -y para q se dibuje apropiadamente
        textSize(8);
        text(x + ',' + y, x + 5, height - y);// 200 -y para q se dibuje apropiadamente
    }

    var root = build_kdtree(data);
    console.log(root);
}

\end{lstlisting}

Tarea 4: Implemente la funcion closest point brute force y naive closest point :
\begin{lstlisting}[style=C]
function closest_point_brute_force(points, point) { 
    var nearPoints = points[0];
	var DistanceMin = distanceSquared(points[0], point);
	for (var i = 1; i < points.length; i++) {
		var t = distanceSquared(points[i], point);
		if (DistanceMin > t) {
			nearPoints = points[i];
			DistanceMin = t;
		}
	}
	return nearPoints;
}
function naive_closest_point(node, point, depth = 0, best = null) {
	if (node != null) {
		var dis = distanceSquared(node.point, point);
		console.log(dis);
		if (best != null && distanceSquared(best, point) < dis) {
			return best;
		}
		else {
			if (node.point[node.axis] > point[node.axis]) {
				return naive_closest_point(node.left, point, depth + 1, node.point);
			}
			else {
				return naive_closest_point(node.right, point, depth + 1, node.point);
			}
		}
	}
	else {
		return best;
	}
}
\end{lstlisting}

Tarea  5:Evalue el resultado de las dos funciones implementadas anteriormente con este conjunto de datos

\begin{lstlisting}[style=C]
vardata=[
  [40,70],
  [70 ,130],
  [90,40],
  [110, 100],
  [140 ,110],
  [160, 100]
  ];
var point=[140 ,90];//query
\end{lstlisting}

Tarea 6: Evalue el resultado de las dos funciones implementadas anteriormente con este conjunto de datos:
\begin{lstlisting}[style=C]
vardata=[
  [40,70],
  [70 ,130],
  [90,40],
  [110, 100],
  [140 ,110],
  [160, 100],
  [150, 30],
  ];
var point=[140 ,90];//query
\end{lstlisting}

Tarea 7: Ahora implemente la funcion closest point, siguiendo las recomendaciones dadas por el docente
\begin{lstlisting}[style=C]
function closest_point(node, point, depth = 0) {

	if (node === null)
		return null;
	var axis = depth % k;
	var next_branch = null;
	var opposite_branch = null;
	if (point[axis] < node.point[axis]) {
		next_branch = node.left;
		opposite_branch = node.right;
	} else {
		next_branch = node.right;
		opposite_branch = node.left;
	}
	var best = closer_point(point, node, closest_point(next_branch, point, depth + 1));
	if (distanceSquared(point, best.point) > Math.abs(point[axis] - node.point[axis])) {
		best = closer_point(point, best, closest_point(opposite_branch, point, depth + 1));
	}

	return best;
}

function closer_point(point, p1, p2) {
	if (p2 == null) {
		return p1;
	}
	var distance = distanceSquared(p1.point, point);
	if (distance < distanceSquared(p2.point, point))
		return p1;
	return p2;
}
\end{lstlisting}

Tarea 8: Averigue e implemente una funcion KNN, que retorna los k puntos mas cercanos a un punto.

\begin{lstlisting}[style=C]
function KNN(points, point, K) {
	var nearPoints = [];
	var found = [];
	for (var i = 0; i < points.length; i++) 
	{
		var aux=distanceSquared(points[i],point);
		nearPoints.push([aux,points[i]])

		nearPoints.sort(function (a,b){
			return a[0]-b[0];
		});
	}
	for(var i = 0; i < nearPoints.length; i++){
		found.push(nearPoints[i].slice(1,2));
	}
	console.log(found.slice(0, K))
}
\end{lstlisting}


Tarea 9: Implemente la funcion range query circle del KD-Tree.
\begin{lstlisting}[style=C]
function range_query_circle(node, center, radio, queue, depth = 0) {
	if (node == null) {
		return null;
	}
	p += 1;
	console.log(p);
	var axis = node.axis;
	var next_branch = null;
	var opposite_branch = null;
	if (center[axis] < node.point[axis]) {
		next_branch = node.left;
		opposite_branch = node.right;
	} else {
		next_branch = node.right;
		opposite_branch = node.left;
	}
	var best = closer_point(center, node, range_query_circle(next_branch, center, radio, queue, depth + 1));
	if (Math.abs(center[axis] - node.point[axis]) <= radio || distanceSquared(center, best.point) > Math.abs(center[axis] - node.point[axis])) {
		if (distanceSquared(center, node.point) <= radio) {
			queue.push(node.point);
		}
		best = closer_point(center, best, range_query_circle(opposite_branch, center, radio, queue, depth + 1));
	}

	return best;
}
\end{lstlisting}

Tarea 10: Implemente la funcion range query rec del KD-Tree, esta vez el range representa un rectangulo.
\begin{lstlisting}[style=C]
function range_query_rect(node , rect , found , depth = 0) {
	if (node === null) {
        return;
    }
    var axis = depth % k;
    if (node.point[axis] < rect[axis][0]){
        range_query_rect(node.right, rect, found, depth+1)
    }
    if (node.point[axis] > rect[axis][1]) {
        range_query_rect(node.left, rect, found, depth+1)
    }
    let x = node.point[0]
    let y = node.point[1]
    if (!(rect[0][0]>x || rect[0][1]<x || rect[1][0]>y || rect[1][1]<y)) {
        found.push(node.point)
    }
    range_query_rect(node.left, rect, found, depth+1)
    range_query_rect(node.right, rect, found, depth+1)
}
\end{lstlisting}

\section{Resultados}



\section{Conclusiones}


	
	
	\end{document}
